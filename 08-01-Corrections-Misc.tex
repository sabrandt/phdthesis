\chapter{Other Corrections}\label{ch:corrs}
\section{Lepton Momentum Corrections}
Reconstructed lepton momenta require additional corrections in data and simulation to ensure agreement of observable distributions. Several sources which are not fully modeled in the simulation, such as detector alignment, reconstruction software, and magnetic field uncertainty are encompassed by these additional corrections. These effects generally apply to both muons and electrons, but the magnitude of the overall correction is much smaller for muons. Scale factors align the maximum of the \zll \mll distribution observed in data to the expected value, while the resolution of lepton momentum reconstruction in simulation is corrected to match data.
\subsection{Electron Energy Scale and Resolution}
Electron energy is corrected at the ECAL cluster energy level. Corrections are derived for separate categories of observables such as $\eta$, and \et, with data further separated by run number. The scale factors for the data correction are derived by fitting a Breit-Wigner function\cite{Breit:1936zzb} convolved with a Crystal-Ball function\cite{Oreglia:1980cs,Skwarnicki:1986xj} to the dilepton invariant mass distribution for \zee events. The scale factor is the difference between the expected dilepton mass and the maximum determined by this fit. Smearing factors are likewise determined by using the simulated \Z boson invariant mass distribution as a probability density function in a maximum likelihood fit, and using the residual to derive a correction factor. Applying this with a Gaussian smearing is sufficient to describe the data for all categories\cite{Khachatryan:2015iwa}.
Systematic uncertainties related to the energy scale and resolution corrections are computed as the difference between locations of the maximum of the \zee \mll distribution from varying electron categorization. 
Scale and resolution corrections are derived using the method described above, from \zee simulation and the 2017H (\sh) single electron trigger dataset. Due to changing detector conditions and calibration, scale corrections are dependent on run number. Scale corrections for the 2017G (\sg) dataset are based on the 2017H Run Number 306936. As these data were taken nearly consecutively, the LHC and detector conditions are sufficiently similar to produce adequate results for events from all categories considered.
The \mll distributions before and after applying electron energy scale corrections are shown in Figure~\ref{fig:lepscale:zee:13} (Figure~\ref{fig:lepscale:zee:5}) for \sh (\sg). 

\begin{figure}[htbp]
\centering
\includegraphics[width=0.49\textwidth]{plots/LepScaleSmear/plotZee13TeV_noCorr/zee_norm.pdf}
\includegraphics[width=0.49\textwidth]{plots/LepScaleSmear/plotZee13TeV_corr/zee_norm.pdf}
\\
\includegraphics[width=0.49\textwidth]{plots/LepScaleSmear/plotZee13TeV_noCorr/zeelog_norm.pdf}
\includegraphics[width=0.49\textwidth]{plots/LepScaleSmear/plotZee13TeV_corr/zeelog_norm.pdf}
\caption{\zee dilepton mass spectrum at \sh, with (right) and without (left) electron energy scale and resolution corrections.}
\label{fig:lepscale:zee:13}
\end{figure}
\begin{figure}[htbp]
\centering
\includegraphics[width=0.49\textwidth]{plots/LepScaleSmear/plotZee5TeV_noCorr/zee_norm.pdf}
\includegraphics[width=0.49\textwidth]{plots/LepScaleSmear/plotZee5TeV_corr/zee_norm.pdf}
\\
\includegraphics[width=0.49\textwidth]{plots/LepScaleSmear/plotZee5TeV_noCorr/zeelog_norm.pdf}
\includegraphics[width=0.49\textwidth]{plots/LepScaleSmear/plotZee5TeV_corr/zeelog_norm.pdf}
\caption{\zee dilepton mass spectrum at \sg, with (right) and without (left) electron energy scale and resolution corrections.}
\label{fig:lepscale:zee:5}
\end{figure}


\subsection{Muon Momentum Corrections}
As with electrons, muon momentum measurements include the need for corrections, though the effect is much smaller than for electrons. Corrections are derived from average lepton \pt and \zmm invariant mass distribution, so that the maximum and width of the \zmm invariant mass in simulation matches data\cite{Bodek:2012id}.  A single set of corrections is used for the entirety of 2017 data-taking. Distributions of \zmm \mll before and after applying muon momentum corrections are shown in Figure~\ref{fig:lepscale:zmm:5} (Figure~\ref{fig:lepscale:zmm:13}) for \sg (\sh).

\begin{figure}[htbp]
\centering
\includegraphics[width=0.49\textwidth]{plots/LepScaleSmear/plotZmm5TeV_noCorr/zmm.pdf}
\includegraphics[width=0.49\textwidth]{plots/LepScaleSmear/plotZmm5TeV_corr/zmm.pdf}
\\
\includegraphics[width=0.49\textwidth]{plots/LepScaleSmear/plotZmm5TeV_noCorr/zmmlog.pdf}
\includegraphics[width=0.49\textwidth]{plots/LepScaleSmear/plotZmm5TeV_corr/zmmlog.pdf}
\caption{\zmm dilepton mass spectrum at \sg, with (right) and without (left) muon Rochester corrections.}
\label{fig:lepscale:zmm:5}
\end{figure}
\begin{figure}[htbp]
\centering
\includegraphics[width=0.49\textwidth]{plots/LepScaleSmear/plotZmm13TeV_noCorr/zmm.pdf}
\includegraphics[width=0.49\textwidth]{plots/LepScaleSmear/plotZmm13TeV_corr/zmm.pdf}
\\
\includegraphics[width=0.49\textwidth]{plots/LepScaleSmear/plotZmm13TeV_noCorr/zmmlog.pdf}
\includegraphics[width=0.49\textwidth]{plots/LepScaleSmear/plotZmm13TeV_corr/zmmlog.pdf}
\caption{\zmm dilepton mass spectrum at \sh, with (right) and without (left) muon momentum corrections.}
\label{fig:lepscale:zmm:13}
\end{figure}


%%%%%%%%%%%%%%%%%%%%%%%%%%%%%%%%%%%%%%%%%%%%
%                Prefiring

\section{ECAL L1 Trigger Prefiring}\label{ch:prefire}

Radiation damage to the $\mathrm{PbWO_4}$ crystals in the ECAL result in color centers forming within the crystal lattice, reducing transparency and altering light propagation through the crystal. Corrections to account for this effect were light transmission were not applied in a way which completely removed the timing drift. Trigger primitives (TPs) in the forward regions of ECAL, $2.0 < |\eta| < 3.0$, were affected by the timing drift during the 2017 data-taking period. TPs in the forward ECAL region could be incorrectly associated with the prior bunch crossing, an effect described as "prefiring". The global trigger rules disallow the collection of consecutive bunch crossings, and no more than one event accepted per three consecutive bunch crossings. Therefore events susceptible to prefiring will be discarded by the trigger while the prior event will be read out.
The prefiring effect is not described in simulation, and rates of prefiring due to large ECAL deposits are studied for jets and photons. Prefiring efficiency scale factors are calculated for an event based on the kinematics of the photons and jets, as described in Equation~\ref{eq:prefiring_weight}.

\begin{figure}[htbp]
\centering
\includegraphics[width=0.49\textwidth]{plots/Prefire/L1prefiring_photonpt_2017G.pdf}
\includegraphics[width=0.49\textwidth]{plots/Prefire/L1prefiring_jetpt_2017G.pdf}
\caption{Pre-firing probability maps for photons (left) and jets (right) for 2017G (\sg). The $z-\mathrm{axis}$ represents the probability of an object with $(\pt,\eta)$ causing pre-firing. Objects with higher \pt and $|\eta| \sim 3$ are more likely to cause pre-firing. Objects with $|\eta| < 2$ do not cause pre-firing.}
\label{fig:prefire:2017G}
\end{figure}

\begin{figure}[htbp]
\centering
\includegraphics[width=0.49\textwidth]{plots/Prefire/L1prefiring_photonpt_2017H.pdf}
\includegraphics[width=0.49\textwidth]{plots/Prefire/L1prefiring_jetpt_2017H.pdf}
\caption{Pre-firing probability maps for photons (left) and jets (right) for 2017H (\sh). The $z-\mathrm{axis}$ represents the probability of an object with $(\pt,\eta)$ causing pre-firing. Objects with higher \pt and $|\eta| \sim 3$ are more likely to cause pre-firing. Objects with $|\eta| < 2$ do not cause pre-firing.}
\label{fig:prefire:2017H}
\end{figure}
\begin{equation}
    \epsilon_{pref} = 1 - P(\mathrm{prefire}) = \prod_{i=\gamma,jets}{(1 - \epsilon_i^{pref}(\eta,p_T))}
    \label{eq:prefiring_weight}
\end{equation}
In cases where a photon and jet overlap (with $\Delta  R < 0.4$), $\epsilon_i=\mathrm{max}(\epsilon_\gamma,\epsilon_{jet})$.
Prefiring rates by $(\pt,\eta)$ for jets and photons are shown in Figure~\ref{fig:prefire:2017G} for 2017G (\sg) and Figure~\ref{fig:prefire:2017H} for 2017H (\sh). These are derived from a tag-and-probe method using the trigger rules to select a set of "un-prefireable" events as a reference value. Per-event prefiring efficiency scale factors are applied to all simulated samples. Correction factors on the \Wpm and \Z boson yields are listed in Table~\ref{tab:prefire:5} for 2017G (\sg) and Table~\ref{tab:prefire:13} for 2017H (\sh). Uncertainties in the prefiring rate per object are taken to be the maximum of the statistical uncertainty or 20\% of the prefiring rate for the particular $(\pt,\eta)$ bin. The effect of prefiring on the \zee rapidity distribution for 2017G (2017H) is shown in  Figure~\ref{fig:prefire:zrap:2017G} (Figure~\ref{fig:prefire:zrap:2017H}). These figures also demonstrate the effectiveness of the corrections at mitigating the effects of prefiring on forward events.
\begin{figure}[htb]
\centering
\includegraphics[width=0.49\textwidth]{plots/Prefire/Zee5_Zrap_noPrefire.png}
\includegraphics[width=0.49\textwidth]{plots/Prefire/Zee5_Zrap_inclPrefire.png}
\caption{\zee rapidity before (left) and after (right) pre-firing corrections, 2017G (\sg). Uncertainty (gray) shows $\pm 1 \sigma$ deviation from central value of prefiring efficiency per event.}
\label{fig:prefire:zrap:2017G}
\end{figure}

\begin{figure}[htb]
\includegraphics[width=0.49\textwidth]{plots/Prefire/Zee13_Zrap_noPrefire.png}
\includegraphics[width=0.49\textwidth]{plots/Prefire/Zee13_Zrap_inclPrefire.png}
\caption{\zee rapidity before (left) and after (right) pre-firing corrections, 2017H (\sh). Uncertainty (gray) shows $\pm 1 \sigma$ deviation from central value of prefiring efficiency per event.}
\label{fig:prefire:zrap:2017H}
\end{figure}
%%%% Table containing prefire effects, 5 TeV
\begin{table}[htbp]
\begin{center}
\scalebox{0.8}{
\begin{tabular}{|c|c|c|c|c|}
\hline
Process & Jets only & Photons only & Jets \& Photons \\\hline \hline
$W\rightarrow e^+\nu$      & 1.00995 & 1.02327 & 1.02524 \\
$W\rightarrow e^-\nu$      & 1.00971 & 1.02257 & 1.02452 \\
$Z\rightarrow ee$          & 1.01438 & 1.04056 & 1.04227 \\
\hline
$W\rightarrow \mu^+\nu$   & 1.01117 & 1.00173 & 1.01214 \\
$W\rightarrow \mu^-\nu$   & 1.01075 & 1.00145 &  1.01156\\
$Z\rightarrow \mu\mu$     & 1.01498 & 1.00117 & 1.01567 \\
\hline
\end{tabular} }
\end{center}


\caption{Correction factors to account for prefiring in the \sg (2017G) dataset. The "Jets \& Photons" column includes proper counting of overlapping objects and is the scale factor used to renormalize the MC.}
\label{tab:prefire:5}
\end{table}

%%%% Table containing prefire effects, 13 TeV
\begin{table}[htbp]
\begin{center}
\scalebox{0.8}{
\begin{tabular}{|c|c|c|c|c|}
\hline
Process & Jets only & Photons only & Jets \& Photons \\\hline \hline
$W\rightarrow e^+\nu$      & 1.01090 & 1.03131 & 1.03436 \\
$W\rightarrow e^-\nu$      & 1.00984 & 1.02716 & 1.02993\\
$Z\rightarrow ee$          & 1.01177 & 1.03696 & 1.0395 \\
\hline
$W\rightarrow \mu^+\nu$   & 1.00979 & 1.00375 & 1.01193 \\
$W\rightarrow \mu^-\nu$   & 1.00887 & 1.00335 & 1.01080 \\
$Z\rightarrow \mu\mu$     & 1.01159 & 1.00304 & 1.01331 \\
\hline
\end{tabular} }
\end{center}


\caption{Correction factors to account for prefiring in the \sh (2017H) dataset. The "Jets \& Photons" column includes proper counting of overlapping objects and is the scale factor used to renormalize the MC.}
\label{tab:prefire:13}
\end{table}
