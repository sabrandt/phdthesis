\chapter{Summary}
Precision measurements of inclusive cross sections in the electron and muon channels at \sg and \sh are made for:
\begin{itemize}
    \item \Wp boson cross section
    \item \Wm boson cross section
    \item \W boson cross section
    \item $\Wp/\Wm$ cross section ratio
    \item \Z boson cross section
    \item $\Wp/\Z$ cross section ratio
    \item $\Wm/\Z$ cross section ratio
    \item $\W/\Z$ cross section ratio
\end{itemize}
These represent the first \W and \Z boson cross section measurements performed at the CMS experiment at \sg. The measurements presented include systematic uncertainties of $< 1\%$ from the measurement and 1.7 \% (1.5\%) at \sh (\sg) from the luminosity calibration. \\
% as well as the ratio of the \sg/\sh cross section measurements. Results are summarized in Table~\ref{tab:final:xs}.
%% generic description
\subsubsection{Recap}
Events were selected from well-identified and isolated electrons and muons. Candidate \Z boson events required two oppositely charged leptons of the same flavor, and \W boson events required the presence of a well-identified and isolated lepton with no other leptons reconstructed in the event.
%% corrections & uncertainties
Lepton reconstruction and identification efficiency corrections were derived by the tag-and-probe method from \zee and \zmm samples. Hadronic recoil corrections were derived from \zmm in data and simulation and \wmunu simulation, to improve \met modeling. Additionally, ECAL L1 prefiring efficiency corrections and lepton momentum scale and resolution corrections were applied. Uncertainties associated with these corrections were propagated from the physical observables to the final discriminant distribution. Uncertainties on theoretical calculations from higher-order QCD, NNLO QCD, NNLL resummation, and NLO EWK are also determined.
%% fitting % results
Final yields and cross section values were determined by performing a fit simultaneously to the \Wp, \Wm, and \Z boson channels. Uncertainties in observable distributions, uncertainties in normalizations, and background process normalizations were correlated in all channels where appropriate.\\
%%%
\subsubsection{Discussion}
The measured cross sections show good agreement between the electron and muon channels at each \s. The measured cross section value for \Wp, \Wm, \W, and \Z boson channels depends on the total integrated luminosity for the dataset, and at the time of writing this thesis the luminosity calibrations for these datasets were not finalized. Therefore, conclusions should not be drawn from comparisons between the inclusive cross sections of the \sg and \sh datasets. However, the cross section ratios $\Wp/\Wm$, $\Wp/\Z$, $\Wp/\Z$, $\W/\Z$ are independent of the luminosity measurement. The ratio measurements are consistent across both lepton channels as well as both \s. 

Precision measurements of the \Wp, \Wm, \W, and \Z boson production cross sections can directly provide insight into some of the proton PDFs as well as setting the foundation for further measurements, particularly differential cross sections, which are important to the global PDF fits. Inclusive cross section and ratio measurements can elucidate some information about the underlying flavor PDFs, for example the ratio of $\Wp/\Wm$ is primarily affected by the relative contributions of the $u$ and $d$ while the $\W/\Z$ ratios are additionally influenced by heavier flavor PDFs.
