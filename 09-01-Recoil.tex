\chapter{Hadronic Recoil Corrections}\label{ch:recoil}

Missing transverse energy spectra of the \W and \Z boson signal simulation does not fully describe the observed distributions in data. This is attributable to multiple effects, such as mis-modeling of multiple scattering and detector simulation, or imperfect detector calibration in data. Therefore, the simulated \met distribution is not able to describe the observed \met distribution to sufficient precision necessary for this measurement. In order to compensate for this, \zmm events in data and simulation are used to produce corrections to the hadronic recoil. These corrections are used to match the \met resolution and response in simulated \W and \Z boson events to that of data.

The \W and \Z boson share a similar production mechanism and are similar in mass, so data corrections derived from \zmm. The hadronic recoil of each event is corrected, and the \met is recomputed based on these corrected values. 

The hadronic recoil is characterized as the negative vector sum of the \met and the \pt of the daughter leptons from the \W or \Z boson, as shown in Equation~\ref{eq:ch10:recoil}. This is effectively everything "else" in the event except for the \W boson and daughter leptons.
\begin{equation}
\vec{U}=-(\vec{E_T^{miss}}+\sum{p_{T}^l})
    \label{eq:ch10:recoil}
\end{equation}

The recoil is split into two components: the projection parallel to the boson momentum, \upar, and perpendicular to the boson momentum, $u_\perp$, as shown in Equations~\ref{eq:ch10:upar} and~\ref{eq:ch10:uprp}. 
\begin{equation}
    u_{||} = ||\vec{U}\cdot \vec{Z_{p_T}}||
    \label{eq:ch10:upar}
\end{equation}

\begin{equation}
    u_\perp = ||\vec{U} -  u_{||}||
    \label{eq:ch10:uprp}
\end{equation}


The \upar and \uprp observables are both the source and target of the recoil corrections. The method by which these distributions are parametrized and corrected are described in the following sections.

%% Also get a figure for this section, probably the vector diagram of the recoils


%%%%%%%%%%%%%%%%%%%%%%%%%%%%%%%%%%%%%%%%%
%%%%%%%%%%%    modeling
%%%%%%%%%%%%%%%%%%%%%%%%%%%%%%%%%%%%%%%%%

\section{Recoil Modeling} \label{ch:recoil:modeling}
The recoil corrections provide a transformation from the simulated recoil distributions into a data-like distribution. The transformation is created from the difference in recoil distributions in \zll samples from simulation and data. Recoil distributions in simulated \W boson events are corrected by the difference between data and simulated \zll recoil.

\section{Parametrization Derivation}
Recoil is parametrized in general the same way for each of the samples:
\begin{enumerate}
\item \upar and \uprp are binned by boson \pt
\item \upar and \uprp  in each \pt bin are fit with a double-Gaussian function
\end{enumerate}
Figures~\ref{fig:recoil:data_fit_example}
 and ~\ref{fig:recoil:mc_fit_example} provide examples of these double-Gaussian fits in data and simulation. Recoil parametrization is done in data for \zmm and simulation for \zmm and \wmunu. The \pt of the \Z bosons is determined by the reconstructed dilepton \pt. For \W bosons, the \pt is taken directly from the generator-level information in the simulation. The parametrization of the recoil in the \W boson sample is further separated by charge, to be applied separately to $W^+$ and $W^-$.
Additionally, fits for \upar and \uprp in Z data include the simulated dilepton background contributions from electroweak and \ttbar sources.

\begin{figure}
\centering
\includegraphics[width=0.49\linewidth]{plots/Recoil/example-data-pfu1fit_12.pdf}
\includegraphics[width=0.49\linewidth]{plots/Recoil/example-data-pfu2fit_12.pdf}
\caption{Examples of the recoil parametrization fits for $u_{parallel}$ (left) and $u_{perp}$ (right) in data for $20~\mathrm{GeV} < m_{\mu\mu}  < 22.5~\mathrm{GeV}$. The two Gaussians are shown, with their sum in blue.}
\label{fig:recoil:data_fit_example}
\end{figure}
\begin{figure}
\centering
\includegraphics[width=0.49\linewidth]{plots/Recoil/example-mc-pfu1fit_12.pdf}
\includegraphics[width=0.49\linewidth]{plots/Recoil/example-mc-pfu2fit_12.pdf}
\caption{Examples of the recoil parametrization fits for $u_{||}$ (left) and $u_{\perp}$ (right) in MC for $20~\mathrm{GeV} < m_{\mu\mu}  < 22.5~\mathrm{GeV}$. The two Gaussians are shown, with their sum in blue.}
\label{fig:recoil:mc_fit_example}
\end{figure}

%%%%%%%%%%%%%%%%%%%%%%%%%%%%%%%%%%%%%%%%%
%%%%%%%%%%%    application
%%%%%%%%%%%%%%%%%%%%%%%%%%%%%%%%%%%%%%%%%

% put in a couple of plots
\section{Application of Corrections}\label{ch:recoil:apply}
The corrections derived in the previous section can be applied to the \W or \Z Monte Carlo. For simulated \W or \Z event, the original \upar and \uprp are replaced by a corrected value and the \met is recomputed. 
\subsubsection{Correcting Recoil} \label{ch10:recoil:apply}
Using the same boson \pt definition that was used to create the parametrization, the recoil distributions in simulated \W and \Z bosons are corrected in the following manner: 
\begin{enumerate}
    \item Each fit result is integrated to create a cumulative distribution function (CDF)
    \item A p-value is determined by evaluating the CDF of the \W boson simulation: $p_{i}=F_{U_{i}}^{M}(u_{i})$
    \item $u_i^{M}$ value is determined by finding the root of the CDF from simulated \Z bosons at $p$: $u_{i}^{M}={F_{U_{i}}^{Data}}^{-1}(p_{i})$
    \item $u_i^{Data}$ value is determined  by finding the root of the CDF from \Z bosons in data at $p$: $u_{i}^{Data}={F_{U_{i}}^{Data}}^{-1}(p_{i})$
    \item The original recoil value is shifted by the difference between the \Z boson recoil in data ($u_i^D$) and simulation ($u_i^M$):  $u_{i}' = u_{i} + (u_i^D - u_i^M) $
    \item $u_{||}'$ and $u_{\perp}'$ are used to compute a new MET value for the event
\end{enumerate}
When applying recoil corrections to \zll events, the first step is based off the \Z boson simulation and the steps effectively reduce to replacing the $u_{i}$ components with the result of step 4. 

\subsubsection{Validation}
The effect of the recoil corrections on the \zmm \met and recoil distributions is shown in Figures~\ref{fig:recoil:validation:met}~and~\ref{fig:recoil:validation:recoil}. Closure of the recoil corrections is performed by comparing the mean and width of the recoil distributions of \zll events in data, simulated \zll without corrections, and \zll with corrections. These are shown in Figure~\ref{fig:recoil:validation:13} [13 TeV] and Figure~\ref{fig:recoil:validation:5} [5 TeV].
\begin{figure}
\centering
\includegraphics[width=0.49\linewidth]{plots/Recoil/close_nocorr_13/fitmetp.png}
\includegraphics[width=0.49\linewidth]{plots/Recoil/close_corr_13/fitmetp.png}
\caption{MET spectrum for Zmm sample, pre- and post-recoil correction.}
\label{fig:recoil:validation:met}
\end{figure}
\begin{figure}
\centering
\includegraphics[width=0.49\linewidth]{plots/Recoil/close_nocorr_13/fitrecoilp.png}
\includegraphics[width=0.49\linewidth]{plots/Recoil/close_corr_13/fitrecoilp.png}
\caption{Recoil spectrum for Zmm sample, pre- and post-recoil correction.}
\label{fig:recoil:validation:recoil}
\end{figure}

\begin{figure}
\centering
\begin{subfigure}{.50\textwidth}
\centering
\includegraphics[width=\linewidth]{plots/Recoil/validation_13/resolution_par.pdf}
 \caption{Resolution of \upar}
%   \label{fig:Eff:el:5TeV:GSFSel:pos
\end{subfigure}%
\centering
\begin{subfigure}{.50\textwidth}
\centering
\includegraphics[width=\linewidth]{plots/Recoil/validation_13/resolution_prp.pdf}
\caption{Resolution of \uprp}
%   \caption{1a}
%   \label{fig:Eff:el:5TeV:GSFSel:pos
\end{subfigure}%
\\
\centering
\begin{subfigure}{.50\textwidth}
\centering
\includegraphics[width=\linewidth]{plots/Recoil/validation_13/response_par.pdf}
\caption{$p_T$-corrected response of \upar}
%   \caption{1a}
%   \label{fig:Eff:el:5TeV:GSFSel:pos
\end{subfigure}%
\caption{Closure of the recoil corrections derived from and applied to $Z\rightarrow\mu\mu$ samples for the 13 TeV dataset. The raw recoil distribution in Monte Carlo (blue) is corrected (red) towards the target spectrum of data (black).}
\label{fig:recoil:validation:13}
\end{figure}

\begin{figure}
\centering
\begin{subfigure}{.50\textwidth}
\centering
\includegraphics[width=\linewidth]{plots/Recoil/validation_5/resolution_par.pdf}
 \caption{Resolution of \upar}
%   \label{fig:Eff:el:5TeV:GSFSel:pos
\end{subfigure}%
\centering
\begin{subfigure}{.50\textwidth}
\centering
\includegraphics[width=\linewidth]{plots/Recoil/validation_5/resolution_prp.pdf}
\caption{Resolution of \uprp}
%   \caption{1a}
%   \label{fig:Eff:el:5TeV:GSFSel:pos
\end{subfigure}%
\\
\centering
\begin{subfigure}{.50\textwidth}
\centering
\includegraphics[width=\linewidth]{plots/Recoil/validation_5/response_par.pdf}
\caption{$p_T$-corrected response of \upar}
%   \caption{1a}
%   \label{fig:Eff:el:5TeV:GSFSel:pos
\end{subfigure}%
\caption{Closure of the recoil corrections derived from and applied to $Z\rightarrow\mu\mu$ samples for the 5 TeV dataset. The raw recoil distribution in Monte Carlo (blue) is corrected (red) towards the target spectrum of data (black).}
\label{fig:recoil:validation:5}
\end{figure}



%%%%%%%%%%%%%%%%%%%%%%%%%%%%%%%%%%%%%%%%%
%%%%%%%%%%%    Sysstematics
%%%%%%%%%%%%%%%%%%%%%%%%%%%%%%%%%%%%%%%%%

\section{Uncertainties}\label{ch:recoil:unc}
The parametrization method described above implicitly makes the assumption that the recoil distribution can uniformly be described by a double-Gaussian over the entire \pt range as well as the assumption that the parametrization can be treated as one inclusive rapidity region. Alternative models are used to address the systematic uncertainty due to these assumptions. 

The impact of these uncertainties is accounted for by propagating the differences in \met through \mt to as uncertainties on the \W boson signal fit as described in the following chapter. Additionally, statistical uncertainties for each of the fit parameters is propagated to the final result in the same way.
The details of how each of the uncertainties is modeled is calculated is listed below.
\begin{enumerate}
    \item \textbf{Binning}: The baseline rapidity ($y$) binning is a single inclusive bin. An alternative using three $y$ bins ($|y|<0.5$,~ $0.5 <  |y| < 1.0$, and $|y| < 1.0$) is fit with the double-Gaussian function. 
    \item \textbf{Model}: The baseline fit model is a double-Gaussian. A Gaussian kernel is used as an alternative. 
    \item \textbf{Statistical uncertainty}: One additional correction set per free parameter is created to estimate the impact of statistical uncertainty in the fit results on the \met modeling. There are a total of 10 correction sets, 6 for $u_{||}$ and 4 for $u_{\perp}$ (two Gaussians with $\mu$,~$\sigma$, and normalization, $u_{\perp}$ has $\mu=0$ fixed for both Gaussians). Principal component analysis of each fit result is used to create the variation correction sets by diagonalizing the covariance matrix and varying one eigenvalue by $\pm 1 \sigma$ and creating a new function from the altered covariance matrix.
\end{enumerate}
Each of these uncertainty models produces an alternate set of recoil corrections. These are applied as described in Section~\ref{ch10:recoil:apply}. The difference in \met from each of the listed configurations is incorporated as an uncertainty on the \W boson signal \mt distribution in the \W signal extraction fit.