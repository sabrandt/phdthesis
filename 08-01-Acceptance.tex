\chapter{Acceptance}\label{ch:acceptance}

\section{Acceptance Calculation}
The acceptance, $A$, for \Wpm and \Z boson events is the fraction of simulated events producing decay products within the fiducial volume satisfying the geometric and kinematic requirements. The specific requirements are $\pt>25\GeV$ and $|\eta| < 2.4$ for both electrons and muons (Table~\ref{tab:Acc:Gen:Cuts}). An additional requirement on transverse mass, $\mt > 40 \GeV$, is
applied for \Wpm events. For \Z events, only those which are generated with \masswindow are considered.
%% Cut Type
\begin{table}[htbp]
\begin{center}
\scalebox{0.8}{
\begin{tabular}{ccc}
\hline
Observable  & Muon  & Electron \\
\hline \hline
$|\eta|$ & < 2.4 &  < 2.4 \\
$p_T$ & > 25 GeV  & > 25 GeV \\
\hline
\end{tabular}}
\end{center}
\caption{Kinematic and fiducial requirements for leptons.}
\label{tab:Acc:Gen:Cuts}
\end{table}
Generator-level acceptance can be computed before and after the effects of final-state radiation (FSR) are simulated. The acceptance values for \Wp, \Wm, and \Z bosons are calculated from \aMCATNLO with NNPDF3.1 PDF and \PYTHIA8.2 for parton showering are shown in Table~\ref{tab:Acc:Gen:Val:5TeV} (Table~\ref{tab:Acc:Gen:Val:13TeV}) for \sg (\sh). The post-FSR column indicates that the leptons have undergone bremsstrahlung, and the dressed lepton category incorporates final state radiation photons into the lepton kinematics. This is done by identifying any photons which are within $\Delta R < 0.1$ of the lepton and adding them back to the lepton momentum vector. Statistical uncertainty in the generator-level acceptance is negligible. Additional uncertainties from theoretical sources are discussed in the next section.
%%%% Table containing the Ele ID+Iso cuts
\begin{table}[htbp]
\begin{center}
\scalebox{0.8}{
\begin{tabular}{|c|c|c|c|c|}
\hline
Process & $A_{Gen}(\mathrm{Post-FSR})$ & $A_{Gen}(\mathrm{Dressed}$) \\\hline \hline
$W\rightarrow e^+\nu$     & 0.535 & 0.556 \\
$W\rightarrow e^-\nu$     & 0.504 & 0.521 \\
$W\rightarrow e\nu$       & 0.523 & 0.542 \\
$Z\rightarrow ee$          & 0.445 & 0.468 \\
\hline
$W\rightarrow \mu^+\nu$   & 0.548 & 0.555 \\
$W\rightarrow \mu^-\nu$   & 0.514 & 0.518 \\
$W\rightarrow \mu\nu$     & 0.535 & 0.541 \\
$Z\rightarrow \mu\mu$     & 0.461 & 0.468 \\
\hline
\end{tabular} }
\end{center}


\caption{Acceptance for post-FSR and dressed leptons at \sg.}
\label{tab:Acc:Gen:Val:5TeV}
\end{table}

%%%% Table containing the Ele ID+Iso cuts
\begin{table}[htbp]
\begin{center}
\scalebox{0.8}{
\begin{tabular}{|c|c|c|c|c|}
\hline
Process & $A_{Gen}(\mathrm{Post-FSR})$ & $A_{Gen}(\mathrm{Dressed}$) \\\hline \hline
$W\rightarrow e^+\nu$     & 0.418 & 0.434 \\
$W\rightarrow e^-\nu$     & 0.434 & 0.449 \\
$W\rightarrow e\nu$       & 0.425 & 0.440 \\
$Z\rightarrow ee$          & 0.360 & 0.378 \\
\hline
$W\rightarrow \mu^+\nu$   & 0.427 & 0.433 \\
$W\rightarrow \mu^-\nu$   & 0.443 & 0.448 \\
$W\rightarrow \mu\nu$     & 0.434 & 0.439 \\
$Z\rightarrow \mu\mu$     & 0.372 & 0.378 \\
\hline
\end{tabular} }
\end{center}
\caption{[13 TeV] Acceptance for post-FSR and dressed leptons}
\label{tab:Acc:Gen:Val:13TeV}
\end{table}

%Undressed Leptons,,,,,
%Channel,Acceptance,PDF unc. (frac),QCD scale unc. (frac),PDF scale unc. (%),QCD scale unc. (%)
%Wm+,0.48469,0.00047,0.00091,0.0473,0.09077
%Wm-,0.56133,0.00035,0.00062,0.03481,0.06199
%Zmm,0.37218,0.00041,0.00252,0.04109,0.25163
%We+,0.45785,0.00048,0.00109,0.04784,0.10909
%We-,0.53222,0.00035,0.00071,0.03479,0.07144
%Zee,0.35955,0.0004,0.00245,0.04031,0.24478
%,,,,,
%,,,,,
%Dressed leptons,,,,,
%Channel,Acceptance,PDF unc. (frac),QCD scale unc. %(frac),PDF scale unc. (%),QCD scale unc. (%)
%Wm+,0.48543,0.00047,0.00091,0.04726,0.09092
%Wm-,0.56201,0.00035,0.00063,0.0348,0.0627
%Zmm,0.37833,0.00041,0.00251,0.04112,0.25149
%We+,0.45943,0.00048,0.00107,0.04774,0.10733
%We-,0.53372,0.00035,0.00073,0.03476,0.07325
%Zee,0.37834,0.0004,0.00242,0.04039,0.24161

%%%%% 
% Acceptance values for samples generated with \aMCATNLO and \POWHEG are shown in Table~\ref{tab:Acc:ele:5TeV} (electrons) and Table~\ref{tab:Acc:mu:5TeV} (muons) for \sg and Table~\ref{tab:Acc:ele:13TeV} (electrons) and Table~\ref{tab:Acc:mu:13TeV} (muons) for \sh. 
% %%%% Table containing the Ele ID+Iso cuts
\begin{table}[htbp]
\begin{center}
\scalebox{1.0}{
\begin{tabular}{|c|c|c|}
\hline
Process & \aMCATNLO & \POWHEG \\\hline \hline
\wep        &  &  \\
\wem        &  &  \\
\wenu         &  &  \\
$\Wp/\Wm$    &  &  \\
\zee         &  &  \\
$\Wp/\Z$   &  &  \\
$\Wm/\Z$     &  &  \\
$\W/\Z$     &  &  \\
\hline
\end{tabular} }
\end{center}

\caption{Acceptance values for electron channels at \serag (2017G).}
\label{tab:Acc:ele:5TeV}
\end{table}

%%%% Table containing the Ele ID+Iso cuts
\begin{table}[htbp]
\begin{center}
\scalebox{1.0}{
\begin{tabular}{|c|c|c|}
\hline
Process & \aMCATNLO & \POWHEG \\\hline \hline
\wep        &  &  \\
\wem        &  &  \\
\wenu         &  &  \\
$\Wp/\Wm$    &  &  \\
\zee         &  &  \\
$\Wp/\Z$   &  &  \\
$\Wm/\Z$     &  &  \\
$\W/\Z$     &  &  \\
\hline
\end{tabular} }
\end{center}

\caption{Acceptance values for muon channels at \serag (2017G).}
\label{tab:Acc:mu:5TeV}
\end{table}
% %%%% Table containing the Ele ID+Iso cuts
\begin{table}[htbp]
\begin{center}
\scalebox{1.0}{
\begin{tabular}{|c|c|c|}
\hline
Process & \aMCATNLO & \POWHEG \\\hline \hline
\wep        &  0.4337 &    \\
\wem        &  0.4493 &   \\
\wenu         &  0.4415 &   \\
$\Wp/\Wm$    &  0.9651 &     \\
\zee         &  0.3783  &      \\
$\Wp/\Z$   &   1.1463  &    \\
$\Wm/\Z$     &  1.1877  &     \\
$\W/\Z$     &   1.1670 &     \\
\hline
\end{tabular} }
\end{center}

\caption{Acceptance values for electron channels at \serah (2017H).}
\label{tab:Acc:ele:13TeV}
\end{table}

%%%% Table containing the Ele ID+Iso cuts
\begin{table}[htbp]
\begin{center}
\scalebox{1.0}{
\begin{tabular}{|c|c|c|}
\hline
Process & \aMCATNLO & \POWHEG \\\hline \hline
\wmp        & 0.4327 &  0.4278 \\
\wmm        & 0.4482 &  0.4402\\
\wmunu         & 0.4404 &  0.4340\\
$\Wp/\Wm$    & 0.9654 &  0.9718  \\
\zmm         & 0.3783  &    0.3797 \\
$\Wp/\Z$   &   1.1436 &  1.1266 \\
$\Wm/\Z$     &  1.1846 &   1.1593 \\
$\W/\Z$     &  1.1641 &   1.1430 \\
\hline
\end{tabular} }
\end{center}

\caption{Acceptance values for muon channels at \serah (2017H).}
\label{tab:Acc:mu:13TeV}
\end{table}

%%%%%%%%%%%%%%%%%%%%%%%%%%%%%%%%%%%%%%%%%%%%%%%%%%%%%%%%
\section{Systematic Uncertainty}\label{ch:acc:unc}
%% Theory Uncertainties

The highest order calculation available for calculating the W cross section is NNLO, and uncertainty comes from the lack of higher order terms. The uncertainty due to these higher order terms is estimated by varying the the QCD renormalization (\mur) and factorization (\muf) scales and evaluating the impact this has on the acceptance. Acceptance is calculated with variations on the \mur and \muf scales of a factor of 2 from their baseline value, and the uncertainty is taken to be the maximum deviation from the baseline acceptance value. Uncertainties are shown in  Table~\ref{tab:Acc:Sys:QCD}. Uncertainties in the NNPDF3.1 sets and $\alpha_s$ are evaluated as the envelope of the differences of each acceptance variation from the baseline acceptance.Differences between acceptance provided by the the MADGRAPH5\_\aMCATNLO and the RESBOS predictions with NNLL accuracy are taken as a systematic uncertainty. Both sets of preditions use CT14 PDF sets for consistency. Modeling of electromagnetic FSR is provided by \PYTHIA and \PHOTOS, with \POWHEG for the matrix-element calculations. Uncertainties in the FSR modeling are estimated as the difference in acceptance between the two models.



%%%% Table containing the $e$ ID+Iso cuts
\begin{table}[htbp]
\begin{center}
\scalebox{0.8}{
\begin{tabular}{|c|c|c|}
\hline
(13 TeV) & $e$[\%] & $\mu$ [\%]\\
\hline \hline
$W^+$     & 0.273 & 0.258 \\
$W^-$     & 0.228 & 0.192 \\
$W$       & 0.254 & 0.228 \\
$Z$       & 0.245 & 0.252 \\
$W^+/W^-$ & 0.052 & 0.091 \\
$W^+/Z$   & 0.149 & 0.157 \\
$W^-/Z$   & 0.127 & 0.122 \\
$W/Z$     & 0.137 & 0.136 \\
\hline
\end{tabular} }
\quad
\scalebox{0.8}{
\begin{tabular}{|c|c|c|}
\hline
(5 TeV) & $e$[\%] & $\mu$ [\%]\\
\hline \hline
$W^+$     & 0.241 & 0.244 \\
$W^-$     & 0.146 & 0.199 \\
$W$       & 0.204 & 0.226 \\
$Z$       & 0.203 & 0.193 \\
$W^+/W^-$ & 0.141 & 0.138 \\
$W^+/Z$   & 0.443 & 0.436 \\
$W^-/Z$   & 0.347 & 0.391 \\
$W/Z$     & 0.406 & 0.418 \\

\hline
\end{tabular} }
\quad
\scalebox{0.8}{
\begin{tabular}{|c|c|c|}
\hline
(13/5 TeV) & $e$ [\%]& $\mu$ [\%]\\
\hline \hline
$W^+$     & 0.419 & 0.396 \\
$W^-$     & 0.234 & 0.228 \\
$W$       & 0.343 & 0.301 \\
$Z$       & 0.187 & 0.180 \\
$W^+/W^-$ & 0.186 & 0.229 \\
$W^+/Z$   & 0.362 & 0.356 \\
$W^-/Z$   & 0.237 & 0.298 \\
$W/Z$     & 0.312 & 0.335 \\
\hline
\end{tabular} }
\end{center}

\caption{Summary of QCD factorization and renormalization scale uncertainties for 13 TeV, 5 TeV, and 13/5 TeV ratio.}
\label{tab:Acc:Sys:QCD}
\end{table}
%%%% Table containing the $e$ ID+Iso cuts
\begin{table}[htbp]
\begin{center}

\scalebox{0.8}{
\begin{tabular}{|c|c|c|}
\hline
(13 TeV) & $e$ & $\mu$ \\
\hline \hline
$W^+$     & 0.039 & 0.039 \\
$W^-$     & 0.035 & 0.035 \\
$W$       & 0.035 & 0.035 \\
$Z$       & 0.040 & 0.041 \\
$W^+/W^-$ & 0.026 & 0.025 \\
$W^+/Z$   & 0.024 & 0.024 \\
$W^-/Z$   & 0.021 & 0.021 \\
$W/Z$     & 0.019 & 0.020 \\
\hline
\end{tabular} }
\quad
\scalebox{0.8}{
\begin{tabular}{|c|c|c|}
\hline
(5 TeV) & $e$ & $\mu$ \\
\hline \hline
$W^+$     & 0.012 & 0.012 \\
$W^-$     & 0.019 & 0.019 \\
$W$       & 0.010 & 0.010 \\
$Z$       & 0.018 & 0.017 \\
$W^+/W^-$ & 0.023 & 0.022 \\
$W^+/Z$   & 0.023 & 0.022 \\
$W^-/Z$   & 0.025 & 0.025 \\
$W/Z$     & 0.021 & 0.021 \\
\hline
\end{tabular} }
\quad
\scalebox{0.8}{
\begin{tabular}{|c|c|c|}
\hline
(13/5 TeV) & $e$ & $\mu$ \\
\hline \hline
$W^+$     & 0.041 & 0.040 \\
$W^-$     & 0.040 & 0.039 \\
$W$       & 0.037 & 0.036 \\
$Z$       & 0.045 & 0.046 \\
$W^+/W^-$ & 0.035 & 0.033 \\
$W^+/Z$   & 0.032 & 0.032 \\
$W^-/Z$   & 0.032 & 0.032 \\
$W/Z$     & 0.027 & 0.027 \\
\hline
\end{tabular} }


\end{center}

\caption{Summary of PDF uncertainties for 13 TeV, 5 TeV, and 13/5 TeV ratio.}
\label{tab:acc:sys:pdf:all}
\end{table}
% comment back in some time
% %%%% Table containing the Ele ID+Iso cuts
\begin{table}[htbp]
\begin{center}
\scalebox{0.8}{
\begin{tabular}{|c|c|c|c|c|}
\hline
 & QCD & PDF & Resummation & EWK \\
\hline \hline
$W^+$     & 0.273 & 0.039 & 0.000 & 0.000 \\
$W^-$     & 0.228 & 0.035 & 0.000 & 0.000 \\
$W$       & 0.254 & 0.035 & 0.000 & 0.000 \\
$Z$       & 0.245 & 0.040 & 0.000 & 0.000 \\
$W^+/W^-$ & 0.052 & 0.026 & 0.000 & 0.000 \\
$W^+/Z$   & 0.149 & 0.024 & 0.000 & 0.000 \\
$W^-/Z$   & 0.127 & 0.021 & 0.000 & 0.000 \\
$W/Z$     & 0.137 & 0.019 & 0.000 & 0.000 \\
\hline
\end{tabular} }
\end{center}
\caption{Uncertainties on electron channel acceptance, 13 TeV.}
\label{tab:Acc:unc:ele:13TeV}
\end{table}


% %%%% Table containing the Ele ID+Iso cuts
\begin{table}[htbp]
\begin{center}
\scalebox{0.8}{
\begin{tabular}{|c|c|c|c|c|}
\hline
 & QCD & PDF & Resummation & EWK \\
\hline \hline
$W^+$     & 0.258 & 0.039 & 0.000 & 0.000 \\
$W^-$     & 0.192 & 0.035 & 0.000 & 0.000 \\
$W$       & 0.228 & 0.035 & 0.000 & 0.000 \\
$Z$       & 0.252 & 0.041 & 0.000 & 0.000 \\
$W^+/W^-$ & 0.091 & 0.025 & 0.000 & 0.000 \\
$W^+/Z$   & 0.157 & 0.024 & 0.000 & 0.000 \\
$W^-/Z$   & 0.122 & 0.021 & 0.000 & 0.000 \\
$W/Z$     & 0.136 & 0.020 & 0.000 & 0.000 \\
\hline
\end{tabular} }
\end{center}
\caption{Uncertainties on muon channel acceptance, 13 TeV.}
\label{tab:Acc:unc:mu:13TeV}
\end{table}


% %%%% Table containing the Ele ID+Iso cuts
\begin{table}[htbp]
\begin{center}
\scalebox{0.8}{
\begin{tabular}{|c|c|c|c|c|}
\hline
 & QCD & PDF & Resummation & EWK \\
\hline \hline
$W^+$     & 0.241 & 0.012 & 0.000 & 0.000 \\
$W^-$     & 0.146 & 0.019 & 0.000 & 0.000 \\
$W$       & 0.204 & 0.010 & 0.000 & 0.000 \\
$Z$       & 0.203 & 0.018 & 0.000 & 0.000 \\
$W^+/W^-$ & 0.141 & 0.023 & 0.000 & 0.000 \\
$W^+/Z$   & 0.443 & 0.023 & 0.000 & 0.000 \\
$W^-/Z$   & 0.347 & 0.025 & 0.000 & 0.000 \\
$W/Z$     & 0.406 & 0.021 & 0.000 & 0.000 \\
\hline
\end{tabular} }
\end{center}


\caption{Uncertainties on electron channel acceptance, 5 TeV.}
\label{tab:Acc:unc:ele:5TeV}
\end{table}


% %%%% Table containing the Ele ID+Iso cuts
\begin{table}[htbp]
\begin{center}
\scalebox{0.8}{
\begin{tabular}{|c|c|c|c|c|}
\hline
 & QCD & PDF & Resummation & EWK \\
\hline \hline
$W^+$     & 0.244 & 0.012 & 0.000 & 0.000 \\
$W^-$     & 0.199 & 0.019 & 0.000 & 0.000 \\
$W$       & 0.226 & 0.010 & 0.000 & 0.000 \\
$Z$       & 0.193 & 0.017 & 0.000 & 0.000 \\
$W^+/W^-$ & 0.138 & 0.022 & 0.000 & 0.000 \\
$W^+/Z$   & 0.436 & 0.022 & 0.000 & 0.000 \\
$W^-/Z$   & 0.391 & 0.025 & 0.000 & 0.000 \\
$W/Z$     & 0.418 & 0.021 & 0.000 & 0.000 \\
\hline
\end{tabular} }
\end{center}


\caption{Uncertainties on muon channel acceptance, 5 TeV.}
\label{tab:Acc:unc:mu:5TeV}
\end{table}




% \input{ch09/tab.09.02.syst/tab.09.02.Acc.Gen.Sys.13.tex}
% \input{ch09/tab.09.02.syst/tab.09.02.Acc.Gen.Sys.5.tex}
% \input{ch09/tab.09.02.syst/tab.09.02.Acc.Gen.Sys.Rat.tex}

%\input{ch09/tab.09.02.syst/tab.09.02.Acc.Gen.All.tex}

%%% Unsure about this part... 
% \section{Acceptance at Reconstruction Level}
% One of the inputs to the cross section calculation is the efficiency-corrected acceptance. This value is determined at the level of full event reconstruction, with the appropriate lepton efficiency scale factors applied to each event. Efficiency scale factors are described in detail in Chapter~\ref{ch7}(Reference Eff. SF chapter). 

% The $A \times \epsilon$ value is computed by running the selection process, 

% \input{ch09/tab.09.03.Acc.Reco/tab.09.03.Reco.13TeV.tex}
% \input{ch09/tab.09.03.Acc.Reco/tab.09.03.Reco.5TeV.tex}
% \input{ch09/tab.09.03.Acc.Reco/tab.09.03.Reco.13to5.tex}

