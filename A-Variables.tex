\chapter{Coordinates \&  Variables}~\label{ch:variables}
This section provides an explanation of the coordinate system conventions and common terminology used in this thesis.

\subsubsection{Coordinate Conventions}
Generally the detector can be though of in cylindrical coordinates, with the $z$-axis aligned to the beamline and $z=0$ at the collision point.
\begin{itemize}
    \item $z$: the $z$-axis is aligned with the beamline through CMS
    \item $r$ radial distance from beamline
    \item 
\end{itemize}

\subsubsection{Kinematic Variables}
\begin{itemize}
    \item $p$
    \item $\pt$
    \item $\eta$
    \item $y = \frac{1}{2}\ln\frac{E+p_z}{E-p_z}$
    \item $\Delta R = \sqrt{(\phi_1-\phi_2)^2 + (\eta_1-\eta_2)^2}$
    
\end{itemize}

$$ y = \frac{1}{2}\ln{\frac{E+p_z}{E-p_z}}  $$

$$\Delta R = \sqrt{(\phi_1-\phi_2)^2 + (\eta_1-\eta_2)^2} $$

$$ \eta = -\ln \tan\bigg(\frac{\theta}{2}\bigg) $$

$$\pt = |\vec{p}| \mathrm{sech}(\eta)  $$