\chapter{Acceptance}\label{ch:acceptance}

\section{Acceptance Calculation}
The acceptance, $A$, for \Wpm and \Z boson events is the fraction of simulated events producing decay products within the fiducial volume satisfying the geometric and kinematic requirements. The specific requirements are $\pt>25\GeV$ and $|\eta| < 2.4$ for both electrons and muons (Table~\ref{tab:Acc:Gen:Cuts}). An additional requirement on transverse mass, $\mt > 40 \GeV$, is
applied for \Wpm events. For \Z events, only those which are generated with \masswindow are considered.
%% Cut Type
\begin{table}[htbp]
\begin{center}
\scalebox{0.8}{
\begin{tabular}{ccc}
\hline
Observable  & Muon  & Electron \\
\hline \hline
$|\eta|$ & < 2.4 &  < 2.4 \\
$p_T$ & > 25 GeV  & > 25 GeV \\
\hline
\end{tabular}}
\end{center}
\caption{Kinematic and fiducial requirements for leptons.}
\label{tab:Acc:Gen:Cuts}
\end{table}
Generator-level acceptance can be computed before and after the effects of final-state radiation (FSR) are simulated. The acceptance values for \Wp, \Wm, and \Z bosons are calculated from \aMCATNLO with NNPDF3.1 PDF and \PYTHIA8.2 for parton showering are shown in Table~\ref{tab:Acc:Gen:Val:5TeV} (Table~\ref{tab:Acc:Gen:Val:13TeV}) for \sg (\sh). The post-FSR column indicates that the leptons have undergone bremsstrahlung, and the dressed lepton category incorporates final state radiation photons into the lepton kinematics. This is done by identifying any photons which are within $\Delta R < 0.1$ of the lepton and adding them back to the lepton momentum vector. Statistical uncertainty in the generator-level acceptance is negligible. Additional uncertainties from theoretical sources are discussed in the next section.
%%%% Table containing the Ele ID+Iso cuts
\begin{table}[htbp]
\begin{center}
\scalebox{0.8}{
\begin{tabular}{|c|c|c|c|c|}
\hline
Process & $A_{Gen}(\mathrm{Post-FSR})$ & $A_{Gen}(\mathrm{Dressed}$) \\\hline \hline
$W\rightarrow e^+\nu$     & 0.535 & 0.556 \\
$W\rightarrow e^-\nu$     & 0.504 & 0.521 \\
$W\rightarrow e\nu$       & 0.523 & 0.542 \\
$Z\rightarrow ee$          & 0.445 & 0.468 \\
\hline
$W\rightarrow \mu^+\nu$   & 0.548 & 0.555 \\
$W\rightarrow \mu^-\nu$   & 0.514 & 0.518 \\
$W\rightarrow \mu\nu$     & 0.535 & 0.541 \\
$Z\rightarrow \mu\mu$     & 0.461 & 0.468 \\
\hline
\end{tabular} }
\end{center}


\caption{Acceptance for post-FSR and dressed leptons at \sg.}
\label{tab:Acc:Gen:Val:5TeV}
\end{table}

%%%% Table containing the Ele ID+Iso cuts
\begin{table}[htbp]
\begin{center}
\scalebox{0.8}{
\begin{tabular}{|c|c|c|c|c|}
\hline
Process & $A_{Gen}(\mathrm{Post-FSR})$ & $A_{Gen}(\mathrm{Dressed}$) \\\hline \hline
$W\rightarrow e^+\nu$     & 0.418 & 0.434 \\
$W\rightarrow e^-\nu$     & 0.434 & 0.449 \\
$W\rightarrow e\nu$       & 0.425 & 0.440 \\
$Z\rightarrow ee$          & 0.360 & 0.378 \\
\hline
$W\rightarrow \mu^+\nu$   & 0.427 & 0.433 \\
$W\rightarrow \mu^-\nu$   & 0.443 & 0.448 \\
$W\rightarrow \mu\nu$     & 0.434 & 0.439 \\
$Z\rightarrow \mu\mu$     & 0.372 & 0.378 \\
\hline
\end{tabular} }
\end{center}
\caption{[13 TeV] Acceptance for post-FSR and dressed leptons}
\label{tab:Acc:Gen:Val:13TeV}
\end{table}

%Undressed Leptons,,,,,
%Channel,Acceptance,PDF unc. (frac),QCD scale unc. (frac),PDF scale unc. (%),QCD scale unc. (%)
%Wm+,0.48469,0.00047,0.00091,0.0473,0.09077
%Wm-,0.56133,0.00035,0.00062,0.03481,0.06199
%Zmm,0.37218,0.00041,0.00252,0.04109,0.25163
%We+,0.45785,0.00048,0.00109,0.04784,0.10909
%We-,0.53222,0.00035,0.00071,0.03479,0.07144
%Zee,0.35955,0.0004,0.00245,0.04031,0.24478
%,,,,,
%,,,,,
%Dressed leptons,,,,,
%Channel,Acceptance,PDF unc. (frac),QCD scale unc. %(frac),PDF scale unc. (%),QCD scale unc. (%)
%Wm+,0.48543,0.00047,0.00091,0.04726,0.09092
%Wm-,0.56201,0.00035,0.00063,0.0348,0.0627
%Zmm,0.37833,0.00041,0.00251,0.04112,0.25149
%We+,0.45943,0.00048,0.00107,0.04774,0.10733
%We-,0.53372,0.00035,0.00073,0.03476,0.07325
%Zee,0.37834,0.0004,0.00242,0.04039,0.24161

%%%%% 
% Acceptance values for samples generated with \aMCATNLO and \POWHEG are shown in Table~\ref{tab:Acc:ele:5TeV} (electrons) and Table~\ref{tab:Acc:mu:5TeV} (muons) for \sg and Table~\ref{tab:Acc:ele:13TeV} (electrons) and Table~\ref{tab:Acc:mu:13TeV} (muons) for \sh. 
% %%%% Table containing the Ele ID+Iso cuts
\begin{table}[htbp]
\begin{center}
\scalebox{1.0}{
\begin{tabular}{|c|c|c|}
\hline
Process & \aMCATNLO & \POWHEG \\\hline \hline
\wep        &  &  \\
\wem        &  &  \\
\wenu         &  &  \\
$\Wp/\Wm$    &  &  \\
\zee         &  &  \\
$\Wp/\Z$   &  &  \\
$\Wm/\Z$     &  &  \\
$\W/\Z$     &  &  \\
\hline
\end{tabular} }
\end{center}

\caption{Acceptance values for electron channels at \serag (2017G).}
\label{tab:Acc:ele:5TeV}
\end{table}

%%%% Table containing the Ele ID+Iso cuts
\begin{table}[htbp]
\begin{center}
\scalebox{1.0}{
\begin{tabular}{|c|c|c|}
\hline
Process & \aMCATNLO & \POWHEG \\\hline \hline
\wep        &  &  \\
\wem        &  &  \\
\wenu         &  &  \\
$\Wp/\Wm$    &  &  \\
\zee         &  &  \\
$\Wp/\Z$   &  &  \\
$\Wm/\Z$     &  &  \\
$\W/\Z$     &  &  \\
\hline
\end{tabular} }
\end{center}

\caption{Acceptance values for muon channels at \serag (2017G).}
\label{tab:Acc:mu:5TeV}
\end{table}
% %%%% Table containing the Ele ID+Iso cuts
\begin{table}[htbp]
\begin{center}
\scalebox{1.0}{
\begin{tabular}{|c|c|c|}
\hline
Process & \aMCATNLO & \POWHEG \\\hline \hline
\wep        &  0.4337 &    \\
\wem        &  0.4493 &   \\
\wenu         &  0.4415 &   \\
$\Wp/\Wm$    &  0.9651 &     \\
\zee         &  0.3783  &      \\
$\Wp/\Z$   &   1.1463  &    \\
$\Wm/\Z$     &  1.1877  &     \\
$\W/\Z$     &   1.1670 &     \\
\hline
\end{tabular} }
\end{center}

\caption{Acceptance values for electron channels at \serah (2017H).}
\label{tab:Acc:ele:13TeV}
\end{table}

%%%% Table containing the Ele ID+Iso cuts
\begin{table}[htbp]
\begin{center}
\scalebox{1.0}{
\begin{tabular}{|c|c|c|}
\hline
Process & \aMCATNLO & \POWHEG \\\hline \hline
\wmp        & 0.4327 &  0.4278 \\
\wmm        & 0.4482 &  0.4402\\
\wmunu         & 0.4404 &  0.4340\\
$\Wp/\Wm$    & 0.9654 &  0.9718  \\
\zmm         & 0.3783  &    0.3797 \\
$\Wp/\Z$   &   1.1436 &  1.1266 \\
$\Wm/\Z$     &  1.1846 &   1.1593 \\
$\W/\Z$     &  1.1641 &   1.1430 \\
\hline
\end{tabular} }
\end{center}

\caption{Acceptance values for muon channels at \serah (2017H).}
\label{tab:Acc:mu:13TeV}
\end{table}

%%%%%%%%%%%%%%%%%%%%%%%%%%%%%%%%%%%%%%%%%%%%%%%%%%%%%%%%
\section{Systematic Uncertainty}\label{ch:acc:unc}
Uncertainties in the measurement due to theoretical predictions are studied by using the acceptance and comparing baseline values to alternate predictive models. The differences are taken as the uncertainties.

\subsubsection{PDF uncertainties}
The proton PDFs describe the momentum probability distributions of quarks and gluons within the proton, and are obtained by fitting to experimental data. Several different collaborations produce PDFs, with derivations using different techniques and different datasets. Methods for evaluating uncertainties in the PDFs due to uncertainties in their parameters are also provided, generally in the form of error PDFs with a $\pm 1 \sigma$ variation on each parameter value. NNPDF provides replica PDFs which are created by using a Monte Carlo technique to sample the probability distributions of observables. Following the procedure outlined in Reference~\cite{Butterworth:2015oua}, the uncertainty in acceptance due to the PDF uncertainties is taken as the standard deviation of the replica set acceptance values. Results are listed in Table~\ref{tab:thyunc:ele:13} and Table~\ref{tab:thyunc:mu:13}.

\subsubsection{Resummation and NNLO QCD}\label{ch:resummation}
For low-\pt boson production, the fixed order calculations are unreliable due to soft gluon emissions producing logarithmic divergences in the cross section calculation
~\cite{Collins:1984kg}. Resummation of the logarithmic terms is included as one of the components of the boson production cross section, and different tools can provide predictions to different orders of accuracy. Parton shower models with leading logarithmic (LL) accuracy (e.g. \PYTHIA, \SHERPA, \HERWIG ~\cite{Sjostrand:2014zea,Gleisberg:2008ta,Bahr:2008pv}) can be combined with fixed-order calculations (e.g. \aMCATNLO, \MINLO, \POWHEG) to provide full event descriptions~\cite{Nason:2004rx,Frixione:2002ik,Alioli:2010xd,Alwall:2014hca}. Higher-order resummation terms can be matched to the fixed-order calculations to provide accurate predictions over the entire boson \pt range~\cite{Balazs:1995nz,Catani:2015vma}. 
The primary simulations are provided by \aMCATNLO, interfaced to \PYTHIA8 for parton showering and \MADGRAPH5 for the hard-scatter matrix element calculations. Resummation to LL is provided by \PYTHIA8 and QCD calculations to NLO are provided by \MADGRAPH5. Higher-order descriptions---NNLO for perturbative QCD and next-to-next-to-leading-logarithmic (NNLL) for resummation of soft QCD effects---are provided by \RESBOS~\cite{Ladinsky:1993zn, Balazs:1997xd, Landry:2002ix} for \sh and DYTURBO using the CT14~\cite{Dulat:2015mca} NNLO PDF set for \sh and \sg as no \RESBOS grids exist yet for \sg. Differences in acceptance predictions between the baseline sample with QCD to NLO and resummation to LL and the DYTURBO prediction with QCD to NNLO and resummation to NNLL are taken as the uncertainties.

%%\RESBOS~\cite{Ladinsky:1993zn, Balazs:1997xd, Landry:2002ix}

%% Theory Uncertainties
\subsubsection{Higher-Order QCD}
Perturbative QCD predictions for \W and \Z boson production are currently only available to NNLO~\cite{Melnikov:2006kv,Catani:2009sm}. To estimate the effect of the missing higher-order terms of the expansion, the renormalization ($\mu_R$) and factorization ($\mu_F$) scales are varied by a factor of 2 from their baseline value, i.e. baseline $\mu = M_{\W}$ with variations $\mu=M_{\W}/2$ and $\mu=2M_{\W}$. Acceptance computed for each combination of $\mu_R$ and $\mu_F$ variations is compared to the baseline acceptance value and the maximum difference is used to estimate the uncertainty due to missing higher-order QCD terms. Results (as a percentage of the acceptance) are shown in Table~\ref{tab:thyunc:ele:13} and Table~\ref{tab:thyunc:mu:13}.
% aMC@NLO:
%  $\textsc{FxFx}$ scheme~\cite{Frederix:2012ps}


\subsubsection{Electroweak Corrections}\label{ch:ewkcorr}
The electroweak corrections include both types of radiative photon corrections---final state radiation(FSR) and higher-order corrections in $\alpha$. Final state radiation is photons originating from the final state leptons. Higher-order electroweak include radiated photons originating from the \W or \Z boson as well as loop corrections. The baseline sample uses \PYTHIA8 which provides electroweak calculations to LO. To estimate the effect of the NLO electroweak terms, \PHOTOS \cite{Golonka:2005pn} is used as a comparison. Because \PHOTOS cannot be interfaced to the \aMCATNLO generator, two additional sets of simulations are produced using \POWHEG as a generator, interfaced with \PHOTOS or \PYTHIA8. The difference in acceptance values between the two simulations is taken to be the uncertainty due to FSR modeling and missing higher-order electroweak terms. 

% EWK calculations NLO, importnat for high boson pT
% ~\cite{Dittmaier:2014qza,Lindert:2017olm}

% 
%%%% Table containing the $e$ ID+Iso cuts
\begin{table}[htbp]
\begin{center}
\scalebox{0.8}{
\begin{tabular}{|c|c|c|}
\hline
(13 TeV) & $e$[\%] & $\mu$ [\%]\\
\hline \hline
$W^+$     & 0.273 & 0.258 \\
$W^-$     & 0.228 & 0.192 \\
$W$       & 0.254 & 0.228 \\
$Z$       & 0.245 & 0.252 \\
$W^+/W^-$ & 0.052 & 0.091 \\
$W^+/Z$   & 0.149 & 0.157 \\
$W^-/Z$   & 0.127 & 0.122 \\
$W/Z$     & 0.137 & 0.136 \\
\hline
\end{tabular} }
\quad
\scalebox{0.8}{
\begin{tabular}{|c|c|c|}
\hline
(5 TeV) & $e$[\%] & $\mu$ [\%]\\
\hline \hline
$W^+$     & 0.241 & 0.244 \\
$W^-$     & 0.146 & 0.199 \\
$W$       & 0.204 & 0.226 \\
$Z$       & 0.203 & 0.193 \\
$W^+/W^-$ & 0.141 & 0.138 \\
$W^+/Z$   & 0.443 & 0.436 \\
$W^-/Z$   & 0.347 & 0.391 \\
$W/Z$     & 0.406 & 0.418 \\

\hline
\end{tabular} }
\quad
\scalebox{0.8}{
\begin{tabular}{|c|c|c|}
\hline
(13/5 TeV) & $e$ [\%]& $\mu$ [\%]\\
\hline \hline
$W^+$     & 0.419 & 0.396 \\
$W^-$     & 0.234 & 0.228 \\
$W$       & 0.343 & 0.301 \\
$Z$       & 0.187 & 0.180 \\
$W^+/W^-$ & 0.186 & 0.229 \\
$W^+/Z$   & 0.362 & 0.356 \\
$W^-/Z$   & 0.237 & 0.298 \\
$W/Z$     & 0.312 & 0.335 \\
\hline
\end{tabular} }
\end{center}

\caption{Summary of QCD factorization and renormalization scale uncertainties for 13 TeV, 5 TeV, and 13/5 TeV ratio.}
\label{tab:Acc:Sys:QCD}
\end{table}
% %%%% Table containing the $e$ ID+Iso cuts
\begin{table}[htbp]
\begin{center}

\scalebox{0.8}{
\begin{tabular}{|c|c|c|}
\hline
(13 TeV) & $e$ & $\mu$ \\
\hline \hline
$W^+$     & 0.039 & 0.039 \\
$W^-$     & 0.035 & 0.035 \\
$W$       & 0.035 & 0.035 \\
$Z$       & 0.040 & 0.041 \\
$W^+/W^-$ & 0.026 & 0.025 \\
$W^+/Z$   & 0.024 & 0.024 \\
$W^-/Z$   & 0.021 & 0.021 \\
$W/Z$     & 0.019 & 0.020 \\
\hline
\end{tabular} }
\quad
\scalebox{0.8}{
\begin{tabular}{|c|c|c|}
\hline
(5 TeV) & $e$ & $\mu$ \\
\hline \hline
$W^+$     & 0.012 & 0.012 \\
$W^-$     & 0.019 & 0.019 \\
$W$       & 0.010 & 0.010 \\
$Z$       & 0.018 & 0.017 \\
$W^+/W^-$ & 0.023 & 0.022 \\
$W^+/Z$   & 0.023 & 0.022 \\
$W^-/Z$   & 0.025 & 0.025 \\
$W/Z$     & 0.021 & 0.021 \\
\hline
\end{tabular} }
\quad
\scalebox{0.8}{
\begin{tabular}{|c|c|c|}
\hline
(13/5 TeV) & $e$ & $\mu$ \\
\hline \hline
$W^+$     & 0.041 & 0.040 \\
$W^-$     & 0.040 & 0.039 \\
$W$       & 0.037 & 0.036 \\
$Z$       & 0.045 & 0.046 \\
$W^+/W^-$ & 0.035 & 0.033 \\
$W^+/Z$   & 0.032 & 0.032 \\
$W^-/Z$   & 0.032 & 0.032 \\
$W/Z$     & 0.027 & 0.027 \\
\hline
\end{tabular} }


\end{center}

\caption{Summary of PDF uncertainties for 13 TeV, 5 TeV, and 13/5 TeV ratio.}
\label{tab:acc:sys:pdf:all}
\end{table}
% comment back in some time
% %%%% Table containing the Ele ID+Iso cuts
% \begin{table}%[htbp]
% \begin{center}
% \scalebox{0.8}{
% \begin{tabular}{|c|c|c|c|c|}
% \hline
%  & QCD & PDF & Resummation & EWK \\
% \hline \hline
% $W^+$     & 0.136 & 0.40 &     &     \\
% $W^-$     & 0.115 & 0.36 &     &     \\
% $W$       & 0.134 & 0.35 &     &     \\
% $Z$       & 0.121 & 0.42 &     &     \\
% $W^+/W^-$ & 0.199 & 0.31 &     &     \\
% $W^+/Z$   & 0.085 & 0.25 &     &     \\
% $W^-/Z$   & 0.215 & 0.22 &     &     \\
% $W/Z$     & 0.107 & 0.18 &     &     \\
% \hline
% \end{tabular} }
% \end{center}


% \caption{Uncertainties on muon channel acceptance, 13 TeV.}
% \label{tab:Acc:unc:mu:13TeV}
% \end{table}

% %%%% Table containing the Ele ID+Iso cuts
% \begin{table}%[htbp]
% \begin{center}
% % \scalebox{0.8}{
% \begin{tabular}{|c|c|c||}
% \hline
%  & Muon & Electron  \\
% \hline \hline
% $W^+$     & 0.019   & 0.333     \\
% $W^-$     & 0.058   & 0.114    \\
% $W$       & 0.060   & 0.214    \\
% $Z$       & 0.096   & 0.275     \\
% $W^+/W^-$ & 0.048   & 0.083     \\
% $W^+/Z$   & 0.031   & 0.414   \\
% $W^-/Z$   & 0.065   & 0.140   \\
% $W/Z$     & 0.011   & 0.297   \\
% \hline
% \end{tabular} 
% \end{center}


% \caption{Electroweak uncertainty.}
% \label{tab:Acc:unc:mu:13TeV}
% \end{table}






%%%%%%%%%%%%%%%%%%%%%%%%%%%%%%%%%%%%%%%%
%%%%% theory uncertainties
%%%%------------------
%%        electron channel
\begin{table}[htbp]
\begin{center}
\begin{tabular}{ccccccccc}
\hline
Source & \Wp & \Wm & \W & $\Wp/\Wm$ & \Z & $\Wp/\Z$ & $\Wm/\Z$ & $\W/\Z$ \\
\hline \hline
%%% Copy and paste below here %%%%%%%%%
QCD & 0.273 & 0.221 & 0.249 & 0.058 & 0.242 & 0.166 & 0.133 & 0.151 \\
PDF & 0.388 & 0.349 & 0.350 & 0.253 & 0.404 & 0.249 & 0.209 & 0.194 \\
Resummation & 0.705 & 0.638 & 0.679 & 0.067 & 0.391 & 0.317 & 0.251 & 0.291 \\
EWK \& FSR & 0.333 & 0.247 & 0.214 & 0.275 & 0.083 & 0.414 & 0.140 & 0.297 \\
\hline \hline
Total [\%] & 0.91 & 0.80 & 0.83 & 0.38 & 0.62 & 0.60 & 0.38 & 0.48 \\



%%%% Copy and paste above here %%%
\hline \hline
\end{tabular}
\end{center}
\caption{Uncertainties (as \% of acceptance) from theory sources for all muon channel measurements at \sh.}
\label{tab:thyunc:ele:13}
\end{table}

%%%%%%%%%%%%%%%%%%%%%%%%%%%%%%%%%%%%%%%%%%%%%%%%%%%%%
%%%%%     muon channel
\begin{table}[htbp]
\begin{center}
\begin{tabular}{ccccccccc}
\hline
Source & \Wp & \Wm & \W & $\Wp/\Wm$ & \Z & $\Wp/\Z$ & $\Wm/\Z$ & $\W/\Z$ \\
\hline \hline
%%% Copy and paste below here %%%%%%%%%

QCD & 0.262 & 0.233 & 0.243 & 0.080 & 0.251 & 0.118 & 0.147 & 0.127 \\
PDF & 0.386 & 0.348 & 0.351 & 0.253 & 0.412 & 0.241 & 0.212 & 0.192 \\
Resummation & 0.637 & 0.638 & 0.639 & 0.002 & 0.379 & 0.258 & 0.260 & 0.260 \\
EWK \& FSR & 0.140 & 0.058 & 0.186 & 0.108 & 0.048 & 0.090 & 0.198 & 0.137 \\
\hline \hline
Total [\%] & 0.80 & 0.76 & 0.79 & 0.28 & 0.61 & 0.38 & 0.41 & 0.37 \\
%%%% Copy and paste above here %%%
\hline \hline
\end{tabular}
\end{center}
\caption{Uncertainties from theory sources (as \% of acceptance value) for muon channels at \sh.}
\label{tab:thyunc:mu:13}
\end{table}


%%%%%%%%%%%%%%%%%%%%%%%%%%%%%%%%%%%%%%%%
%%%%% theory uncertainties
%%%%------------------
%%        electron channel
\begin{table}[htbp]
\begin{center}
\begin{tabular}{ccccccccc}
\hline
Source & \Wp & \Wm & \W & $\Wp/\Wm$ & \Z & $\Wp/\Z$ & $\Wm/\Z$ & $\W/\Z$ \\
\hline \hline
%%% Copy and paste below here %%%%%%%%%

QCD & 0.267 & 0.226 & 0.249 & 0.048 & 0.242 & 0.147 & 0.112 & 0.130 \\
PDF & 0.039 & 0.036 & 0.036 & 0.026 & 0.040 & 0.024 & 0.021 & 0.019 \\
Resummation & 1.200 & 1.000 & 1.100 & 1.600 & 0.800 & 1.200 & 0.400 & 0.500 \\
EWK & 0.333 & 0.058 & 0.214 & 0.275 & 0.083 & 0.414 & 0.140 & 0.297 \\
 \\
\hline \hline
Total [\%] & 1.27 & 1.03 & 1.15 & 1.62 & 0.84 & 1.28 & 0.44 & 0.59 \\

%%%% Copy and paste above here %%%
\hline \hline
\end{tabular}
\end{center}
\caption{Uncertainties (as \% of acceptance) from theory sources for all muon channel measurements at \sg.[waiting on grid samples for final resummation estimate]}
\label{tab:thyunc:ele:5}
\end{table}

%%%%%%%%%%%%%%%%%%%%%%%%%%%%%%%%%%%%%%%%%%%%%%%%%%%%%
%%%%%     muon channel
\begin{table}[htbp]
\begin{center}
\begin{tabular}{ccccccccc}
\hline
Source & \Wp & \Wm & \W & $\Wp/\Wm$ & \Z & $\Wp/\Z$ & $\Wm/\Z$ & $\W/\Z$ \\
\hline \hline
%%% Copy and paste below here %%%%%%%%%
QCD & 0.266 & 0.190 & 0.228 & 0.114 & 0.251 & 0.155 & 0.110 & 0.130 \\
PDF & 0.039 & 0.036 & 0.034 & 0.025 & 0.042 & 0.024 & 0.022 & 0.020 \\
Resummation & 1.700 & 1.810 & 1.100 & 1.800 & 0.900 & 1.300 & 0.600 & 0.500 \\
EWK & 0.019 & 0.114 & 0.060 & 0.096 & 0.048 & 0.031 & 0.065 & 0.011 \\
 \\
\hline \hline
Total [\%] & 1.72 & 1.82 & 1.12 & 1.81 & 0.94 & 1.31 & 0.61 & 0.52 \\
%%%% Copy and paste above here %%%
\hline \hline
\end{tabular}
\end{center}
\caption{Uncertainties from theory sources (as \% of acceptance value) for muon channels at \sg.[waiting on samples for final resummation estimate]}
\label{tab:thyunc:mu:5}
\end{table}


% %%%% Table containing the Ele ID+Iso cuts
\begin{table}[htbp]
\begin{center}
\scalebox{0.8}{
\begin{tabular}{|c|c|c|c|c|}
\hline
 & QCD & PDF & Resummation & EWK \\
\hline \hline
$W^+$     & 0.244 & 0.012 & 0.000 & 0.000 \\
$W^-$     & 0.199 & 0.019 & 0.000 & 0.000 \\
$W$       & 0.226 & 0.010 & 0.000 & 0.000 \\
$Z$       & 0.193 & 0.017 & 0.000 & 0.000 \\
$W^+/W^-$ & 0.138 & 0.022 & 0.000 & 0.000 \\
$W^+/Z$   & 0.436 & 0.022 & 0.000 & 0.000 \\
$W^-/Z$   & 0.391 & 0.025 & 0.000 & 0.000 \\
$W/Z$     & 0.418 & 0.021 & 0.000 & 0.000 \\
\hline
\end{tabular} }
\end{center}


\caption{Uncertainties on muon channel acceptance, 5 TeV.}
\label{tab:Acc:unc:mu:5TeV}
\end{table}




% \input{ch09/tab.09.02.syst/tab.09.02.Acc.Gen.Sys.13.tex}
% \input{ch09/tab.09.02.syst/tab.09.02.Acc.Gen.Sys.5.tex}
% \input{ch09/tab.09.02.syst/tab.09.02.Acc.Gen.Sys.Rat.tex}

%\input{ch09/tab.09.02.syst/tab.09.02.Acc.Gen.All.tex}

%%% Unsure about this part... 
% \section{Acceptance at Reconstruction Level}
% One of the inputs to the cross section calculation is the efficiency-corrected acceptance. This value is determined at the level of full event reconstruction, with the appropriate lepton efficiency scale factors applied to each event. Efficiency scale factors are described in detail in Chapter~\ref{ch7}(Reference Eff. SF chapter). 

% The $A \times \epsilon$ value is computed by running the selection process, 

% \input{ch09/tab.09.03.Acc.Reco/tab.09.03.Reco.13TeV.tex}
% \input{ch09/tab.09.03.Acc.Reco/tab.09.03.Reco.5TeV.tex}
% \input{ch09/tab.09.03.Acc.Reco/tab.09.03.Reco.13to5.tex}

