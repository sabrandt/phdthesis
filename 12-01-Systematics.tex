\chapter{Systematic Uncertainties}\label{ch:systematics}
%% This chapter should be one big section with tables covering the systematic uncertainties. Point out that the methodology for determining each uncertainty was already provided in the respective chapter

\section{Overview}
Systematic uncertainties which are taken into account in the simultaneous fit of \Wp, \Wm, and \Z are briefly described. Uncertainties in kinematic observables such as lepton \pt or \met which additionally impact the \mt and \mll distributions are included as shape uncertainties. Detailed descriptions were provided in the respective chapters, and the uncertainties are summarized briefly here.

\subsubsection{Luminosity}
The luminosity uncertainty for datasets used in this analysis is $4.8\%$ for \sg dataset and $1.7\%$ for the \sh dataset. 

\subsubsection{Lepton Efficiency}
Potential bias in measuring the lepton reconstruction and identification efficiency is estimated by varying the modeling of signal and backgrounds used in the efficiency  fits. Signal models constructed from dilepton \mll distributions using different generator and final-state radiation programs are compared. Additionally an alternative background modeling function is used in the fits. Tag selection criteria for the tag-and-probe is included by varying the tag selection requirements. These uncertainties are correlated between \Wp, \Wm, and \Z channels for a given lepton flavor. Statistical uncertainties are considered uncorrelated in \pt and $\eta$ bins and efficiency categories, and are also uncorrelated in the \Wp,\Wm, and \Z channels.

\subsubsection{Lepton Momentum Scale}
The lepton \pt is varied by the uncertainty in the lepton momentum scale corrections. The variations in lepton \pt are propagated to the \mt and \mll observables. This is treated as correlated.

\subsubsection{ECAL Pre-Firing}
The uncertainty in the pre-firing probability for an object is $20\%$ of the pre-firing probability or the statistical uncertainty in the correction factor, whichever is more. This is treated as fully correlated.

\subsubsection{\met Modeling}
Bias due to assumptions made in modeling hadronic recoil are estimated by an alternative model. Assumptions about rapidity binning are accounted for with corrections derived in 3 $|y|$ bins and model choice is accounted for by using a Gaussian kernel fitting function. Statistical uncertainties are evaluated by creating correction sets through a principal component analysis with one uncertainty per free parameter. Variants of \mt distributions are treated as fully correlated in the \Wp and \Wm channels.

\subsubsection{Background Modeling}
A shape uncertainty due to the QCD background selection in the \W channels is derived from differences in \mt distributions in the $0.30 < \mathrm{Iso} <  0.45$ and $0.45 < \mathrm{Iso} <  0.60$ control regions. Normalization uncertainties are applied to the \ttbar and diboson backgrounds in the \Z channel and additionally to the Drell-Yann background in the \W channels.

\subsubsection{Theoretical Uncertainties}
Variations in theoretical predictions of cross sections are evaluated for different soucres---QCD factorization ($\mu_F$) and renormalization scale ($\mu_R$), PDF variations, resummation schemes, and final-state radiation models. Uncertainties from QCD scale variations and PDF uncertainties are treated as correlated, while the FSR modeling uncertainties are uncorrelated.

\section{Summary}
Summaries of uncertainties are listed in Table~\ref{tab:syst:mu:13}  and Table~\ref{tab:syst:ele:13} for the \sh muon and electron channels. Table~\ref{tab:syst:mu:5} and Table~\ref{tab:syst:ele:5} contain summaries of the muon and electron channel uncertainties at \sg.



%%%% Tables for EleGSFSel Efficiency  %%%%%
%% Efficiency table for ZeeGSFSel Combined
\begin{table}%[htbp]
\begin{center}
\scalebox{1.0}{
\begin{tabular}{ccccccccc}
\hline
Source & \Wp & \Wm & \W & $\Wp/\Wm$ & \Z & $\Wp/\Z$ & $\Wm/\Z$ & $\W/\Z$ \\
\hline \hline
%%% Copy and paste below here %%%%%%%%%
Lepton reco \& ID & 0.57 & 0.57 & 0.62 & 0.33 & 0.97 & 0.64 & 0.64 & 0.62 \\
Background modeling & 0.08 & 0.10 & 0.09 & 0.02 & 0.02 & 0.09 & 0.11 & 0.10 \\
Recoil modeling & 0.08 & 0.07 & 0.07 & 0.04 & 0.03 & 0.07 & 0.06 & 0.06 \\
Prefire & 0.27 & 0.25 & 0.26 & 0.02 & 0.31 & 0.06 & 0.08 & 0.07 \\
\hline
Total Meas.  & 0.62 & 0.62 & 0.66 & 0.32 & 1.01 & 0.63 & 0.64 & 0.62 \\
\hline
Theory Unc.  & 2.58 & 2.74 & 1.69 & 2.71 & 1.40 & 1.96 & 0.92 & 0.77 \\
\hline
Luminosity  & 4.80 & 4.80 & 4.80 & 0.04 & 4.80 & 0.13 & 0.18 & 0.15 \\
\hline \hline
Total [\%] & 5.48 & 5.56 & 5.13 & 2.73 & 5.10 & 2.06 & 1.13 & 1.00 \\

%%%% Copy and paste above here %%%
\hline \hline
\end{tabular}}
\end{center}
\caption{Systematic uncertainties for all muon channel measurements at \sg.}
\label{tab:syst:mu:5}
\end{table}

%%%% Tables for EleGSFSel Efficiency  %%%%%
%% Efficiency table for ZeeGSFSel Combined
\begin{table}%[htbp]
\begin{center}
\begin{tabular}{ccccccccc}
\hline
Source & \Wp & \Wm & \W & $\Wp/\Wm$ & \Z & $\Wp/\Z$ & $\Wm/\Z$ & $\W/\Z$ \\
\hline \hline
%%% Copy and paste below here %%%%%%%%%
Lepton Reco \& ID  & 0.50 & 0.50 & 0.36 & 0.7 & 0.53 & 0.61 & 0.61 & 0.66\\
Charge Mis-ID  & 0.06 & 0.08 & - & 0.02 & 0.18 & 0.12 & 0.10 & 0.18\\
Background modeling & 0.32 & 0.37 & 0.37 & 0.35 & 0.31 & 0.27 & 0.35 & 0.24 \\
Recoil modeling & 0.12 & 0.13 & 0.10 & 0.06 & 0.17 & 0.10 & 0.15 & 0.12 \\
Prefire & 0.49 & 0.47 & 0.47 & 0.02 & 0.74 & 0.29 & 0.32 & 0.30 \\
\hline
Total Meas.  & 0.78 & 0.78 & 0.69 & 0.72 & 0.98 & 0.74 & 0.79 & 0.78 \\
\hline
Theory Unc.  & 0.49 & 0.49 & 0.42 & 0.39 & 0.31 & 0.66 & 0.50 & 0.56 \\
\hline
Luminosity  & 3.50 & 3.50 & 3.50 & 0.00 & 3.50 & 0.00 & 0.00 & 0.00 \\
\hline \hline
Total [\%] & 3.62 & 3.62 & 3.59 & 0.82 & 3.65 & 0.99 & 0.93 & 0.96 \\
%%%% Copy and paste above here %%%
\hline \hline
\end{tabular}
\end{center}
\caption{Systematic uncertainties for all electron channel measurements at \sg.}
\label{tab:syst:ele:5}
\end{table}

%%%% Tables for EleGSFSel Efficiency  %%%%%
%% Efficiency table for ZeeGSFSel Combined
\begin{table}%[htbp]
\begin{center}
\scalebox{1.0}{
\begin{tabular}{ccccccccc}
\hline
Source & \Wp & \Wm & \W & $\Wp/\Wm$ & \Z & $\Wp/\Z$ & $\Wm/\Z$ & $\W/\Z$ \\
\hline \hline
%%% Copy and paste below here %%%%%%%%%
Lepton Reco \& ID  & 0.35 & 0.33 & 0.28 & 0.45 & 0.39 & 0.51 & 0.51 & 0.49\\
Background modeling & 0.12 & 0.13 & 0.13 & 0.13 & 0.12 & 0.14 & 0.15 & 0.13 \\
Recoil modeling & 0.06 & 0.07 & 0.07 & 0.02 & 0.04 & 0.03 & 0.04 & 0.03 \\
Prefire & 0.25 & 0.23 & 0.23 & 0.02 & 0.28 & 0.04 & 0.06 & 0.05 \\
\hline
Total Meas.  & 0.44 & 0.40 & 0.36 & 0.46 & 0.48 & 0.52 & 0.53 & 0.50 \\
\hline
Theory Unc.  & 0.80 & 0.76 & 0.79 & 0.28 & 0.61 & 0.38 & 0.41 & 0.37 \\
\hline
Luminosity  & 1.70 & 1.70 & 1.70 & 0.00 & 1.70 & 0.00 & 0.00 & 0.00 \\
\hline \hline
Total [\%] & 1.93 & 1.90 & 1.91 & 0.53 & 1.87 & 0.64 & 0.66 & 0.62 \\
%%%% Copy and paste above here %%%
\hline \hline
\end{tabular}}
\end{center}
\caption{Systematic uncertainties for all muon channel measurements at \sh.}
\label{tab:syst:mu:13}
\end{table}

%%%% Tables for EleGSFSel Efficiency  %%%%%
%% Efficiency table for ZeeGSFSel Combined
\begin{table}%[htbp]
\begin{center}
\scalebox{1.0}{
\begin{tabular}{ccccccccc}
\hline
Source & \Wp & \Wm & \W & $\Wp/\Wm$ & \Z & $\Wp/\Z$ & $\Wm/\Z$ & $\W/\Z$ \\
\hline \hline
%%% Copy and paste below here %%%%%%%%%
Lepton Eff.  & 0.50 & 0.50 & 0.36 & 0.7 & 0.53 & 0.61 & 0.61 & 0.66\\
Charge Mis-ID  & - & - & - & - & - & - & - & -\\
Background modeling & 0.32 & 0.34 & 0.29 & 0.29 & 0.32 & 0.23 & 0.26 & 0.19 \\
Recoil modeling & 0.13 & 0.12 & 0.10 & 0.14 & 0.15 & 0.03 & 0.15 & 0.08 \\
Prefire & 0.55 & 0.47 & 0.51 & 0.07 & 0.66 & 0.14 & 0.22 & 0.18 \\
\hline
Total Meas.  & 0.81 & 0.77 & 0.69 & 0.70 & 0.91 & 0.66 & 0.70 & 0.70 \\
\hline
Theory Unc.  & 1.27 & 1.03 & 1.15 & 1.62 & 0.84 & 1.28 & 0.44 & 0.59 \\
\hline
Luminosity  & 1.70 & 1.70 & 1.70 & 0.00 & 1.70 & 0.02 & 0.03 & 0.02 \\
\hline \hline
Total [\%] & 2.27 & 2.13 & 2.16 & 1.76 & 2.10 & 1.44 & 0.83 & 0.91 \\
%%%% Copy and paste above here %%%
\hline \hline
\end{tabular}}
\end{center}
\caption{Systematic uncertainties for all electron channel measurements at \sh.}
\label{tab:syst:ele:13}
\end{table}
